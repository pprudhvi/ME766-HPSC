%----------------------------------------------------------------------------------------
%	PACKAGES AND OTHER DOCUMENT CONFIGURATIONS
%----------------------------------------------------------------------------------------

\documentclass{article}

\usepackage{fancyhdr} % Required for custom headers
\usepackage{lastpage} % Required to determine the last page for the footer
\usepackage{extramarks} % Required for headers and footers
\usepackage[usenames,dvipsnames]{color} % Required for custom colors
\usepackage{graphicx} % Required to insert images
\usepackage{listings} % Required for insertion of code
\usepackage{libertine} % Required for the courier font
\usepackage{inconsolata}
% Margins
\topmargin=-0.45in
\evensidemargin=0in
\oddsidemargin=0in
\textwidth=6.5in
\textheight=9.0in
\headsep=0.25in

\linespread{1.1} % Line spacing

% Set up the header and footer
\pagestyle{fancy}
\lhead{\hmwkAuthorName} % Top left header
\chead{\hmwkClass : \hmwkTitle} % Top center head
\rhead{\firstxmark} % Top right header
\lfoot{\lastxmark} % Bottom left footer
\cfoot{} % Bottom center footer
\rfoot{Page\ \thepage\ of\ \protect\pageref{LastPage}} % Bottom right footer
\renewcommand\headrulewidth{0.4pt} % Size of the header rule
\renewcommand\footrulewidth{0.4pt} % Size of the footer rule

\setlength\parindent{0pt} % Removes all indentation from paragraphs

%----------------------------------------------------------------------------------------
%	CODE INCLUSION CONFIGURATION
%----------------------------------------------------------------------------------------

\definecolor{lineno}{rgb}{0.5,0.5,0.5}
\definecolor{code}{rgb}{0,0.1,0.6}
\definecolor{keyword}{rgb}{0.5,0.1,0.1}
\definecolor{titlebox}{rgb}{0.85,0.85,0.85}
\definecolor{download}{rgb}{0.8,0.1,0.5}
\definecolor{title}{rgb}{0.4,0.4,0.4}
\definecolor{mygrey}{gray}{0.96}

\lstset{
	language=C,
	basicstyle=\ttfamily\small\color{code},
	showspaces=false,
	%showstringspaces=false,
	%numbers=left,
	%firstnumber=1,
	%stepnumber=1,
%	numberfirstline=true,
	%numberstyle=\color{lineno}\sffamily\scriptsize,
	keywordstyle=\color{keyword}\bfseries,
	stringstyle=\itshape,
	%morekeywords={dosync,if},
	deletekeywords={alter},
	%backgroundcolor=\color{mygrey},
	commentstyle=\color{Green},
}

%----------------------------------------------------------------------------------------
%	DOCUMENT STRUCTURE COMMANDS
%	Skip this unless you know what you're doing
%----------------------------------------------------------------------------------------

% Header and footer for when a page split occurs within a problem environment
\newcommand{\enterProblemHeader}[1]{
	\nobreak\extramarks{#1}{#1 continued on next page\ldots}\nobreak
	\nobreak\extramarks{#1 (continued)}{#1 continued on next page\ldots}\nobreak
}

% Header and footer for when a page split occurs between problem environments
\newcommand{\exitProblemHeader}[1]{
	\nobreak\extramarks{#1 (continued)}{#1 continued on next page\ldots}\nobreak
	\nobreak\extramarks{#1}{}\nobreak
}

\setcounter{secnumdepth}{0} % Removes default section numbers
\newcounter{homeworkProblemCounter} % Creates a counter to keep track of the number of problems

\newcommand{\homeworkProblemName}{}
\newenvironment{homeworkProblem}[1][Problem \arabic{homeworkProblemCounter}]{ % Makes a new environment called homeworkProblem which takes 1 argument (custom name) but the default is "Problem #"
	\stepcounter{homeworkProblemCounter} % Increase counter for number of problems
	\renewcommand{\homeworkProblemName}{#1} % Assign \homeworkProblemName the name of the problem
	\section{\homeworkProblemName} % Make a section in the document with the custom problem count
	\enterProblemHeader{\homeworkProblemName} % Header and footer within the environment
}{
	\exitProblemHeader{\homeworkProblemName} % Header and footer after the environment
}

\newcommand{\problemAnswer}[1]{ % Defines the problem answer command with the content as the only argument
\noindent\framebox[\columnwidth][c]{\begin{minipage}{0.98\columnwidth}#1\end{minipage}} % Makes the box around the problem answer and puts the content inside
}

\newcommand{\homeworkSectionName}{}
\newenvironment{homeworkSection}[1]{ % New environment for sections within homework problems, takes 1 argument - the name of the section
	\renewcommand{\homeworkSectionName}{#1} % Assign \homeworkSectionName to the name of the section from the environment argument
	\subsection{\homeworkSectionName} % Make a subsection with the custom name of the subsection
	\enterProblemHeader{\homeworkProblemName\ [\homeworkSectionName]} % Header and footer within the environment
}{
	\enterProblemHeader{\homeworkProblemName} % Header and footer after the environment
}

%----------------------------------------------------------------------------------------
%	NAME AND CLASS SECTION
%----------------------------------------------------------------------------------------

\newcommand{\hmwkTitle}{Assignment\ \#2} % Assignment title
%\newcommand{\hmwkDueDate}{Monday,\ January\ 1,\ 2012} % Due date
\newcommand{\hmwkClass}{ME 766} % Course/class
%\newcommand{\hmwkClassTime}{10:30am} % Class/lecture time
%\newcommand{\hmwkClassInstructor}{Jones} % Teacher/lecturer
\newcommand{\hmwkAuthorName}{110070039} % Your name

%----------------------------------------------------------------------------------------
%	TITLE PAGE
%----------------------------------------------------------------------------------------

\title{
	\vspace{2in}
	\textmd{\textbf{\hmwkClass:\ \hmwkTitle}}\\
	%\normalsize\vspace{0.1in}\small{Due\ on\ \hmwkDueDate}\\
	%\vspace{0.1in}\large{\textit{\hmwkClassInstructor\ \hmwkClassTime}}
	\vspace{3in}
}

\author{\textbf{\hmwkAuthorName}}
\date{} % Insert date here if you want it to appear below your name

%----------------------------------------------------------------------------------------

\begin{document}


%----------------------------------------------------------------------------------------
%	PROBLEM 1
%----------------------------------------------------------------------------------------

% To have just one problem per page, simply put a \clearpage after each problem

The basic code block that does the computation can be written in two ways
as shown in the table below. The code on right side is faster than that on
the left column of the table as it is cache-friendly. It might not make
much difference for small values of \verb+NC+ but for larger values 
the speedup is upto 2. We are going to use this code for matrix multiplication
in all files.
\begin{table}[h]

	\begin{center}
	\begin{tabular}{| l | l |}

			\hline 
			\begin{lstlisting}
for( i=0; i<NR; i++ ){
  for( j=0; j<NC; j++ ){
    for( k=0; k<NC; k++){
	  C[i][j] += A[i][k]*B[k][j];
}
}
}
			\end{lstlisting} 
			&
			\begin{lstlisting}
for( i=0; i<NR; i++ ){
  for( k=0; k<NC; k++){
    for( j=0; j<NC; j++ ){
	  C[i][j] += A[i][k]*B[k][j];
}
}
}
			\end{lstlisting} \\ 
			\hline

		\end{tabular}
	\end{center}
	\caption{Matrix Multiplication; right is cache-friendly}

\end{table}

\begin{table}[h]
	\begin{center}
	\begin{tabular}{| r | r | r |}
		\hline
		N & a & b \\ \hline
		100&0.008075 & 0.008205 \\ \hline
		200&0.063293&0.064621 \\ \hline
		400&0.594145&0.610361 \\ \hline
		800&5.386019&4.080133 \\ \hline
		1000&9.487167&5.573937 \\ \hline
		2000&87.209641&45.544228 \\ \hline
		5000&1908.666260&703.574890 \\ \hline
		8000&8362.995117&2878.790771 \\ \hline
		10000&17790.554688&6591.408691 \\ \hline
	\end{tabular}
\end{center}
\end{table}
\textbf{Results}

\begin{table}[h]

	\begin{center}
		\begin{tabular}{| r | l | l | l | l | l | l | l | l | l |}
			\hline
			N&serial & omp n=2& omp n=4& omp n=8& omp n= 24 & mpi n=2& mpi n=4& mpi n=8& mpi n=24 \\ \hline
			100&0.008205&0.009146&0.002916&0.001719
			&0.006868 & 0.004521 & 0.004978 &0.002792&0.005075\\ \hline
			
			200&0.064621&0.006868&0.041666&	0.021512&0.010956
			&0.058968&0.033004&0.018016&0.022578\\ \hline
		
			400 &0.610361&0.329870&0.186483&0.105709&0.080594 
			&0.345524&0.250767&0.183066&0.092229\\ \hline
		
			800&4.080133&2.195364&1.172719&0.573523&0.486476
			&2.328038&1.368429&0.696170&0.588597\\ \hline
			
			1000&5.573937&2.841520&2.338277&0.804072&0.593531
			&2.898817&2.535994&1.165923&0.717748\\ \hline
			
			2000&45.544228&22.061188&11.366979&5.903165&4.064920
			&22.534018&11.835779&10.047218&4.581032\\ \hline

			5000&703.574890&342.611450&175.706894&90.065071&62.474705
			&346.948578&179.307938&91.601669&57.712799\\ \hline
			
			8000&2878.790771&1400.647705&697.816223&363.321991 
			&234.618332&1434.890625&715.344727&636.908081
			&233.626678 \\ \hline
			
			10000&6591.408691&2737.574951&1798.366455&740.366760 
			&499.029694&2812.980469&1680.102051&877.236450&557.603271\\ 
			\hline
		\end{tabular}
	\end{center}


	\caption{Time taken. N is size of matrix, n is no. of threads/procesess}
\end{table}


\end{document}
